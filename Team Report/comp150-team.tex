% Please do not change the document class
\documentclass{scrartcl}

% Please do not change these packages
\usepackage[hidelinks]{hyperref}
\usepackage[none]{hyphenat}
\usepackage{setspace}
\usepackage{graphicx}
\doublespace

% You may add additional packages here
\usepackage{amsmath}

% Please include a clear, concise, and descriptive title
\title{Teamwork Review}

% Please do not change the subtitle
\subtitle{COMP150 - Teamwork Review}

% Please put your student number in the author field
\author{1507290}

\begin{document}

\maketitle

\abstract{}

\section{Introduction}
Our team experienced issues with maintaining a consistent level of communication throughout the project and assessing whether the scope of the project was appropriate.

\section{Weaknesses that Affected the Collaborative Project}

\subsection{Maintaining a Consistent Level Of Communication}
There were many occasions on which the team did not communicate for several weeks. I attempted to ensure that everyone updated each other about what they were working on and encouraged people to ask for help if stuck. However, this became difficult to maintain as it felt as though I was always initiating communication, making me become stressed and lose motivation. Communication appeared to become less frequent as the project progressed. The lack of communication within the team meant that team members remained stuck on programming problems for extended periods of time, resulting in reduced productivity. When working in a team in the games industry, communication throughout the project is vital to ensure that the project remains on track and progresses at a consistent pace.

Attempting to ask as an individual how the members of the team are doing was an ineffective means of communication, as people have a tendency to say that they are doing fine when asked directly. Instead, a more structured and routine approach to maintaining consistent levels of communication is necessary. In the next collaborative project, I will attempt to avoid this problem by encouraging the team to engage in daily stand-up meetings. This will ensure that the team is communicating every day and finds out what everyone is working on and what issues they are having. The effectiveness of this solution can be evaluated throughout the project by keeping a diary of when the team participated in daily stand-up meetings, as well as at the end of the project by seeing how many user stories were completed.

\subsection{Assessing the Scope of the Project}
The project ended up being overscoped, resulting in most of the game's planned features being dropped in order to have a functional game. I believe there are several issues that contributed to this.  Firstly, three members of the team claimed important game components for Coding Task 2. This meant that other team members felt unable to step in if a critical component had not been delivered in the sprint it was expected, resulting in aspects of the game that relied on that component remaining unimplemented. Secondly, we should have rescoped the project earlier when it became evident we were not going to complete everything. This way, we may have been less stressed and more motivated by an achievable goal. Finally, regular communication would have allowed problems to be overcome more rapidly as well as provided opportunities to discuss rescoping the project.

Ensuring that the scope of a project is appropriate is important in the industry in order to have a functional product ready for deadlines.
In future collaborative projects, we should review the scope and progress of the project at the end of each sprint. If it appears that the project is not progressing as initially expected, we should consider reprioritising the backlog or simplifying the design. Additionally, we should implement user stories that are key components first. The success of this practice can be measured by seeing whether a functional game was produced and whether the design needed to be simplified in order to achieve that.


\section{Conclusion}
Hopefully, applying these practices will allow future collaborative projects to be more organised and successful.

\bibliographystyle{ieeetran}

\end{document}
